\documentclass[sigconf]{acmart}

%%
%% \BibTeX command to typeset BibTeX logo in the docs
\AtBeginDocument{%
  \providecommand\BibTeX{{%
    Bib\TeX}}}



\setcopyright{acmcopyright}
\copyrightyear{2023}
\acmYear{2023}
% \acmDOI{XXXXXXX.XXXXXXX}

%% These commands are for a PROCEEDINGS abstract or paper.
\acmConference[MVA 2023]{MVA 2023}{December
  2023}{Paris, FR}

\begin{document}

%%
%% TITLE
\title{Review: Generative classifiers robustness to Adversarial Attacks?}

%% AUTHORS
\author{Mathis Embit}
\affiliation{
  \institution{ENS Paris-Saclay}
  \city{Saclay}
  \country{France}
}
\email{mathis.embit@ens-paris-saclay.fr}

\author{David Sahna}
\affiliation{
  \institution{ENS Paris-Saclay}
  \city{Saclay}
  \country{France}
}
\email{david.sahna@ens-paris-saclay.fr}

\author{Balthazar Neveu}
\affiliation{
  \institution{ENS Paris-Saclay}
  \city{Saclay}
  \country{France}
}
\email{balthazar.neveu@ens-paris-saclay.fr}


\renewcommand{\shortauthors}{MVA et al.}
%% ABSTRACT
\begin{abstract}
  We'll review the paper "Are Generative Classifiers More Robust to Adversarial Attacks?" \cite{li2019}
\end{abstract}

%% KEYWORDS
\keywords{Adversarial, Attacks, Generative, Deep, Bayes, Robust}


\received{22 December 2023}
% \received[revised]{22 December 2023}
% \received[accepted]{55 December 2023}


%% MAIN DOCUMENT.
\maketitle

\section{Introduction}


\cite{li2019}
$sqrt(a*x+b)$

\section{Context analyzis}

\bibliographystyle{ACM-Reference-Format}
\bibliography{references}
\end{document}
\endinput
